\section{Conclusion}

The results of the testing completed on the system demonstrates its effectiveness at performing the desired estimations. In addition to the surprisingly high tolerance of the system to having a small number of data points and high levels of noise to work with, it should be noted that the ability to perform these estimations precisely for a sparse dataset is at its optimum when at the very least, the dataset has more measurements taken around the early stage in the experimental procedure.

Overall, the practical implications of the development of this system are significant in the field of pharmacokinetics. It should significantly help with the process of drug discovery with respect to drugs that exhibit non-linear pharmacokinetics by allowing the easy identification of the order of the pharmacokinetic dynamics. Additionally, the bayesian approach which has been utilised also has the potential to provide significant improvements to patient care through the ability to utilise prior information regarding the patient and the distributions of parameters associated with their demographics to achieve narrower credibility intervals.